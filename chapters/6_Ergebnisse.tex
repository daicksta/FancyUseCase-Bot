% Dieses Kapitel bildet zusammen mit dem folgenden Kapitel Diskussion den Hauptteil der Arbeit. In übersichtlicher Gliederung und sinnvoller Reihenfolge wird dargestellt, was mittels der eingesetzten Methodik im Hinblick auf die Zielsetzung herausgefunden werden konnte. Die Strukturierung des Kapitels orientiert sich an der Theorie und Methodik. Mit Hilfe von Grafiken und Diagrammen (siehe 2.8.4) werden die Ergebnisse wertfrei dargestellt. Eine Interpretation erfolgt an dieser Stelle noch nicht.
\chapter{Ergebnisse}
	Das auf Basis der vorangegangenen Überlegungen entwickelte Gerät soll in den folgenden Kapiteln anhand technischer Zeichnungen näher erläutert werden.
	Eine vollständige technische Dokumentation ist in \cref{sec:zeichnungen} zu finden.
	\section{Gesamtbauteil}
	\section{Schultergelenk}
	\section{Ober-/Unterarm}
	Die Konstruktion der Ober- und Unterarme wurde zunächst an den Prinzipskizzen wie in \crefrange{fig:prinzipskizze-gesamtansicht}{fig:prinzipskizze-ellbogen} zu sehen orientiert erstellt.
	Dazu wurden entsprechende Bauteile in CAD gezeichnet und als Baugruppe zusammengefügt.
	Allerdings zeigte sich nach ersten Berechnungen zu den künstlichen Muskeln, dass diese Konstruktion deutlich zu schwer war um sie mit der Kraft der künstlichen Muskeln zu bewegen.
	Nach weiteren Recherchen wurde außerdem klar, dass die Bauteile einen extrem hohen Preis in der Fertigung aufwiesen.
	Aus diesen Gründen musste nach kurzer Zeit die Konstruktion komplett überarbeitet werden.
	Dabei wurde besonders darauf geachtet, dass sich das Gewicht möglichst gering hält als auch die Kosten durch die Verwendung von Kaufteilen reduzieren.
	Dies führte dazu, dass die Konstruktion nahezu komplett aus Aluminiumrohren und 3D-geruckten Teilen konstruiert wurde.
	Dabei liegt auch der Vorteil, dass nicht alle Bauteile eigenständig konstruiert werden müssen.
	So lassen sich zum Beispiel die Klemmen, die die Rohrteile miteinander verbinden als gewöhnliche Fahrradklemmen ausführen.
	Auch die Anschlüsse der künstlichen Muskeln können als 3D-gedruckte Teile ausgearbeitet werden und mit einer einfachen Rohrschelle an die Arme befestigt werden.
	Das Ellbogengelenk ist auch mit gekauften und gedruckten Teilen erstellt.
	Dabei wurden alle Teile, die zugekauft werden können auch zugekauft.
	Spezielle Teile wie die Einsätze der Rohre wurden eigenständig konstruiert. 
	\section{Hand}
	Das Handgelenk sowie die Hand bestehen im Wesentlichen aus gekauften Teilen.
	Dabei wurde das Handgelenk so ausgelenkt, dass es sowohl mit einer Klemme mit einer Kugelform als Anschluss ausgestattet werden kann als auch mit einer Schraube der Größe M5 befestigt werden kann.
	Dabei ist lediglich zu beachten, dass die Kugelform mit einem Durchmesser von 28 mm ausgeführt ist.
	Die Klemme bzw. Hand sollte des Weiteren Breit genug ausgearbeitet sein um dem Queue auch eine entsprechende seitliche und vertikale Stabilisation ermöglicht. 
	\section{Künstliche Muskeln}