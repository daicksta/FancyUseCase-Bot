% In diesem Kapitel werden die Ergebnisse hinsichtlich der Problem- und Zielstellung ausgewertet und interpretiert. Die Bedeutung der Ergebnisse für die wissenschaftliche Diskussion und die Konsequenzen für die entsprechenden Praxisfelder sollte argumentiert werden. Die Kapitel Ergebnisse und Diskussion sollten in etwa zwei Drittel Ihrer Arbeit ausmachen.
\chapter{Diskussion}
	Nach der Präsentation der Konstruktionsergebnisse kann nun eine Auswertung über den Erfolg oder die Probleme erfolgen.
	Es sollen ebenfalls auf Aspekte der allgemeinen Vorgehensweise und der Gruppenarbeit eingegangen werden.\par \medskip
	Die gewählten Lösungen bringen alles mit sich, was sie in unserer Vorstellung versprachen.\\
	Das Schultergelenk hat mit dem Zykloidalgetriebe den mechanischen Ansprüchen genügt.
	Da bei dem Akt eines Billardstoßes die Schulter die Arme nur nach vorne und hinten bewegt, und sie nicht etwa nach außen klappt, ist die konstruierte Schulter dem eines echten Menschen beim Stoßen einer Kugel mittels eines Queues relativ nahe.\\
	Der Ober- und Unterarm sind der Ähnlichkeit eines menschlichen Körpers willen als zylindrische Rohre ausgeführt.
	Das bedeutet zugleich aber auch, dass die Form zum konstruieren unpraktischer ist, da in die Rohre auch Schlitze und Löcher eingesetzt werden müssen.
	Die Handhabung mit flachen, rechteckigen Bauteilen wären im Allgemeinen einfacher gewesen.
	Da dies jedoch kein größeres Problem darstellt, ist die ursprüngliche Option beibehalten.\\
	Ähnlich wie beim Schultergelenk sieht es auch mit dem Ellbogengelenk aus.
	Da das Gelenk nur die Aufgabe hat, Ober- und Unterarm miteinander zu verbinden, sie stabil zu halten und richtig auszulenken, genügt ein scharnierartiges Konstrukt den Anforderungen.
	Dadurch, dass bei der geforderten Bewegung der Unterarm nicht nach innen geklappt werden muss, ist der von uns konstruierte Ellbogen mit seiner Bewegungskoordination ausreichend.\\
	Was die Muskeln anbelangt, wurde bereits argumentiert, dass die Ausführung als künstliche Muskeln der menschlichen Anatomie und Physiologie sehr nahekommt und deshalb verwendet wurde.
	Viel einfacher, sowohl für die Konstruktion als auch für die Handhabung, wäre es gewesen, eine alternative Variante zu wählen.
	Dadurch, dass die Muskeln pneumatisch funktionieren, sind sie auch an einem dementsprechenden Antrieb gebunden, welches den Aufwand erhöht.
	Nach Abwägung und auf Wunsch des Kunden bzw. Auftragsgebers ist dennoch auf die künstlichen Muskeln zurückgegriffen worden.\\
	Die Hand schien wie erwartet den kleinsten konstruktiven Aufwand zu verlangen.
	Da sie nur die Aufgabe des Queuehalters hat und dies nicht viel erfordert, kann die Halterung als Kaufteil erworben werden.
	Dann bleibt nur noch das Handgelenk übrig.
	Das Handgelenk verbindet mit der ausgewählten Variante den Unterarm und die Hand beweglich, so dass keine ungewollten Verdrehungen oder Sonstiges auftreten.\par \medskip
	Da zum ganzen Konstruktionsprojekt ein Team in Form einer Dreiergruppe gehört, wird auch dieser Aspekt zur Bewertung herangezogen.\\
	Das Konstruktionsteam kennt sich untereinander relativ gut, weshalb Probleme und Unstimmigkeiten schnell ausdiskutiert werden konnten.
	Des Weiteren ist jedem klar, welches Repertoire jeweils die anderen besitzen, weshalb die Aufgabenverteilung mehr oder weniger vorbestimmt war.\\
	Es gab außerdem wöchentliche Meetings mit dem Auftragsgeber bzw. dem Veranstaltungsleiter, der zu jedem Schritt mit seiner Erfahrung und Expertise Ratschläge geben konnte, die der Gruppe sehr geholfen hat.
	Es konnten somit Fragen und Unklarheiten geklärt und weitere Schritte besprochen werden.\\
	Ein Problem war allerdings die unterschiedlich vielen Know-Hows der Teammitglieder.
	Dies führte dazu, dass an der einen oder anderen Stelle einige Ausarbeitungsideen nicht sofort nachvollzogen werden konnten.
	Eine unterschiedliche Verteilung der Arbeitslast auf die Individuen ließ sich deshalb nicht verhindern.
