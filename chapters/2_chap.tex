\chapter{Funktionsgliederung/-Struktur}
	Um das Gerät strukturiert konstruieren zu können, werden die einzelnen Funktionen, welche der Arm erfüllen soll, gegliedert.

\section{Funktionsbaum}
	Zunächst werden die allgemeinen Funktionen in Kategorien unterteilt, um eine Übersicht zu erschaffen. Diese sind in einem Funktionsbaum (s. \cref{fig:funktionsbaum}) zu sehen.

	\begin{figure}[h]
		\centering
		\includegraphics[width=\textwidth]{"Abb/Funktionsbaum"}
		\caption[Funktionsbaum]{Funktionsbaum}
		\label{fig:funktionsbaum}
	\end{figure}


\section{Morphologische Kästen}
	Im nächsten Schritt sind die erforderlichen Funktionen und mögliche Lösungen in morphologische Kästen (s. \crefrange{fig:morphologische-kasten-teilprobleme}{fig:morphologische-kasten-feedback-system}) aufgestellt. Die von uns gewählten Lösungen sind in grün markiert.\\
	Die Begründungen zu den Lösungsauswahlen werden in \cref{Begründung Lösungsauswahl} näher erläutert.

	\begin{table}[h]
		\centering
		\caption[Morphologischer Kasten der Teilprobleme]{Morphologischer Kasten der Teilprobleme}
		\centering
		\includegraphics[width=\textwidth]{"Abb/Morphologischer Kasten Teilprobleme"}
		\label{fig:morphologische-kasten-teilprobleme}
	\end{table}

	\begin{table}[h]
		\caption[Morphologischer Kasten der Schulter]{Morphologischer Kasten der Schulter}
		\centering
		\includegraphics[width=\textwidth]{"Abb/Morphologischer Kasten Schulter"}
		\label{fig:morphologische-kasten-schulter}
	\end{table}

	\begin{table}[h]
		\caption[Morphologischer Kasten des Ellbogens]{Morphologischer Kasten des Ellbogens}
		\centering
		\includegraphics[width=\textwidth]{"Abb/Morphologischer Kasten Ellbogen"}
		\label{fig:morphologische-kasten-ellbogen}
	\end{table}

	\begin{table}[h]
		\caption[Morphologischer Kasten des Handgelenks]{Morphologischer Kasten des Handgelenks}
		\centering
		\includegraphics[width=\textwidth]{"Abb/Morphologischer Kasten Handgelenk"}
		\label{fig:morphologische-kasten-handgelenk}
	\end{table}

	\begin{table}[h]
		\caption[Morphologischer Kasten des Feedback-Systems]{Morphologischer Kasten des Feedback-Systems}
		\centering
		\includegraphics[width=\textwidth]{"Abb/Morphologischer Kasten Feedback-System"}
		\label{fig:morphologische-kasten-feedback-system}
	\end{table}

\section{Prinzipskizze}
	Da die Funktionen nun definiert sind, kann das Prinzip des Gerätes schematisch in einer Prinzipskizze aufgezeigt werden.\\
	\cref{fig:prinzipskizze-gesamtansicht} zeigt die Hauptbauteile, \cref{fig:prinzipskizze-ellbogen} den Anschluss zwischen Ober- und Unterarm und \cref{fig:prinzipskizze-handgelenk-und-hand} die Hand und ihre Verbindung zum Unterarm.

	\begin{figure}[h]
		\centering
		\includegraphics[width=\textwidth]{"Abb/Prinzipskizze Gesamtansicht"}
		\caption[Prinzipskizze - Gesamtansicht]{Prinzipskizze - Gesamtansicht}
		\label{fig:prinzipskizze-gesamtansicht}
	\end{figure}

	\begin{figure}[h]
		\centering
		\includegraphics[width=\textwidth]{"Abb/Prinzipskizze Ellbogen"}
		\caption[Prinzipskizze - Ellbogen]{Prinzipskizze - Ellbogen}
		\label{fig:prinzipskizze-ellbogen}
	\end{figure}

	\begin{figure}[h]
		\centering
		\includegraphics[width=\textwidth]{"Abb/Prinzipskizze Handgelenk und Hand"}
		\caption[Prinzipskizze - Handgelenk und Hand]{Prinzipskizze - Handgelenk und Hand}
		\label{fig:prinzipskizze-handgelenk-und-hand}
	\end{figure}