% Im Theorieteil ist der aktuelle Stand der Technik/Forschung wertungsfrei darzustellen. Er dient dazu, den wissenschaftlichen Kontext für die Aufgabenstellung herauszuarbeiten. Dementsprechend sollte die Darstellungsweise nicht zu allgemein gewählt werden, sondern nur die für die Aufgabenstellung relevanten Theorien und notwendigen Definitionen dargelegt werden. Stehen mehrere Theorien zur Verfügung, ist ein Überblick zu geben. Definitionen sind nur dann notwendig, wenn sie nicht ohnehin als allgemein, d.h. von Vertreter des Fachbereichs, bekannt vorausgesetzt werden können und für den weiteren Verlauf der Arbeit relevant sind. Die Theorie hat einen entscheidenden Einfluss auf die Methodenwahl und die Interpretation der Ergebnisse. Der Stand der Technik ist u.a. dokumentiert in Monografien, Vortragsmanuskripten, Patenten und in besonderer Weise in Artikeln wissenschaftlicher Zeitschriften. Diese Quellen Seite 5 sind zu nutzen. Es wird eine Analyse von mindestens vier Zeitschriftenartikeln erwartet, von denen mindestens einer in englischer Sprache verfasst ist. Bei der Auswahl der Literatur ist auf Aktualität der Arbeiten zu achten.
\chapter{Theorie}
		Um einschätzen zu können, mit welchen Größenordnungen man es zu tun hat, wird zunächst die Geschwindigkeit berechnet, die beim Ausführen eines typischen Stoßes von einem Billardspieler erreicht wird.
		Da zu typischen Stoßgeschwindigkeiten keine Literaturwerte zu finden waren, kann nach Auswertung verschiedener Videoaufzeichnungen und Sichtung einschlägiger Foren mit guter Sicherheit von einer durchschnittlichen Geschwindigkeit der Kugel von \SI{40}{\kilo\metre\per\hour} bzw. \SI{11,1}{\metre\per\second} ausgegangen werden. 
		Betrachtet man nun die Impulsübertragung ergibt sich, dass der Impuls identisch sein muss. 
		Für den Impuls gilt die Formel.
		\begin{equation}
			p_{Arm} = p_{Kugel}
			\label{eq:Impuls}
		\end{equation}
		Dabei gilt, dass der Impuls mit Masse multipliziert mit der Geschwindigkeit gegeben ist.  
		\begin{equation}
			m\cdot v = m_{Kugel} \cdot v_{Kugel}%
			\label{eq:ImpulsKugel}
		\end{equation}
		Hier ist die Masse des Arms \(m\), die Bewegungsgeschwindigkeit des Arms \(v\), die Masse der Kugel \(m_{Kugel}\) und die Bewegungsgeschwindigkeit der Kugel \(v_{Kugel}\).
		Damit kann die Bewegungsgeschwindigkeit an der Hand des Armes berechnet werden.
		\begin{equation}
			v = \frac{m_{Kugel} \cdot v_{Kugel}}{m}%
			\label{eq:SollgeschwindikeitArm}
		\end{equation}
